\documentclass{article}
\usepackage{amsmath, amsthm, amssymb}
\usepackage{listings}
\usepackage{graphicx}
\usepackage{float}
\usepackage{enumerate}
\usepackage{fancyhdr}
\usepackage[labelfont=bf]{caption}
\usepackage[left=0.75in, top=1in, right=0.75in, bottom=1in]{geometry}
\pagestyle{plain}
\begin{document}
\rhead{Aaron Okano, Anatoly Torchinsky, Samuel Huang, Justin Maple \\ 
      ECS 132: Homework 2}
\thispagestyle{fancy}

% Let the homework begin!
\section*{Problem 1} 

\subsection*{(a)}

Since this is not an infinite series, we cannot use identities derived in the
book. Instead, we must do our own. We must solve for $c$ in
$\displaystyle\sum\limits_{k=1}^{10} ck^2 = 1.0$.
\begin{eqnarray*}
  \sum\limits_{k=1}^{10} ck^2 &=& c \cdot \frac{10(10 + 1)(2\cdot10+1)}{6} \\
  &=& c\cdot 385 \\
  &\Rightarrow& c = \frac{1}{385}
\end{eqnarray*}

\subsection*{(b)}

We can find the expected value of X by computing $\sum\limits_{k=1}^{10} k\cdot
p_{X}(k)$. We know $p_{X}(k) = c\cdot k^2$ and $c = \frac{1}{385}$ so,
\begin{eqnarray*}
  \frac{1}{385}\cdot\sum\limits_{k=1}^{10} k\cdot k^2 &=& \frac{1}{385}
  \sum\limits_{k=1}^{10} k^3 \\
  &=& \frac{1}{385} \cdot \left[ \frac{1}{4}\cdot10^4 + \frac{1}{2}\cdot10^3 +
  \frac{1}{4}\cdot10^2\right] \\
  &\approx& 7.86
\end{eqnarray*}

\subsection*{(c)}

By (3.30), we can compute the variance of X by finding
\begin{eqnarray*}
  \left[\sum\limits_{k=1}^{10} k^2\cdot p_{X}(k) \right] - [EX]^2 &=&
  \frac{1}{385}\cdot \sum\limits_{k=1}^{10} k^3 - 7.86^2 \\
  &=& \frac{1}{385}\left[ \frac{1}{5}\cdot10^5 + \frac{1}{2}\cdot10^4 +
  \frac{1}{3}\cdot10^3 - \frac{1}{30}\cdot10\right] - 7.86^2 \\
  &\approx& 4.02
\end{eqnarray*}

\section*{Problem 2}

First, we take note that each set of coin tosses follows a binomial
distribution. Therefore, we can express $P( X_2 = i ), i = 0,1,2$ as $P( X_2 =
i | C_1 \cup X_2 = i | C_2 ) = P( X_2 = i | C_1 ) + P( X_2 = i | C_2 ) = P(
C_1) P( X_2 = i | C_1 ) + P( C_2 )P( X_2 = i | C_2 )$. This follows
intuitively, since there are two possibilities: one where you pick the
head-weighted coin and one where you pick the tail-weigted coin, after which
the set of tosses is modeled with a binomial distribution. Using this
information, we can construct a pmf for the coin tosses:
\begin{equation*}
  p_{X_2}(k) = 0.5\cdot\binom{2}{k}(0.9)^k(0.1)^{2 - k} +
  0.5\cdot\binom{2}{k}(0.1)^k(0.9)^{2 - k}
\end{equation*}
Plugging this into the handy formula $EX_2 = \sum\limits_{k=0}^{k=2} k\cdot
p_{X_2}(k)$ yields
\begin{eqnarray*}
  EX_2 &=& \sum\limits_{k=0}^{k=2}
  k\cdot\left[0.5\cdot\binom{2}{k}(0.9)^k(0.1)^{2 - k} +
  0.5\cdot\binom{2}{k}(0.1)^k(0.9)^{2 - k}\right] \\
  &=& 0 + \left[(0.5)(2)(0.9)(0.1) + (0.5)(2)(0.1)(0.9)\right] +
  2\cdot\left[(0.5)(0.9)^2 + (0.5)(0.1)^2\right] \\
  &=& 0.18 + 0.82 \\
  &=& 1
\end{eqnarray*}
Finding variance is trivial at this point, since the final term of $EX_2$ can
simply be doubled to give us $E({X_2}^2) = 0.18 + 1.64 = 1.82$. Then $Var(X_2) =
E({X_2}^2) - (EX)^2 = 1.82 - 1^2 = 0.82$.

\section*{Problem 3}

\subsection*{(a)}

We can split this problem into a few special cases. First, we know that any
time the state transitions from any number of packets in the buffer to one
packet in the buffer, the container can pass through with at least as many open
spaces as packets currently in the buffer. This can be generalized to the
following: if there is no new packet, the container must pass through holding
no more than $c - i$ packets or if a new packet is generated, it must pass
through with no more than $c - ( i + 1 )$ packets. In equation form, this is
\begin{equation} 
 p_{i0} = (1 - p)\cdot P( N \leq c - i ) + p\cdot P( N \leq c - (i + 1) ),
 \quad i=0,1,\ldots,b-1
\end{equation}

Second, we also know that when the buffer is full, whether or not a new packet
is generated, the state can only make a successful transition if the container
has exactly enough space for the number of packets to be removed. This means we
only need to take into account the probability that the container has exactly
that much room. Mathematically, this reduces to
\begin{equation}
  p_{bj} = P( N = c - (b - j) ), \quad j = 1,2,\ldots,b
\end{equation}
Furthermore, if $b = c$, then we can include $j = 0$ in the range of this
function.

Every remaining case we can generalize much like our first case, with the
exception that the number of space available in the container must be
\emph{exactly} the amount necessary to switch from the current state to the
next. Once again, as an equation this is
\begin{equation}
  p_{ij} = (1 - p)\cdot P( N = c - (i - j) ) + p\cdot P( N = c - [(i + 1) - j]
  ), \quad i=0,1,\ldots,b-1 \quad j=1,2,\ldots,b
\end{equation}
Note that $P( N > c ) = 0$.

Knowing that $N$ is uniformly distributed on the range (0,5) allows us to
reduce equation (2) to
\begin{equation}
  p_{bj} = \frac{1}{c + 1}
\end{equation}
and equation (3) to
\begin{align}
  p_{ij} &= (1 - p)\cdot\frac{1}{c + 1} + p\cdot\frac{1}{c + 1} \nonumber \\
         &= \frac{1}{c + 1}
\end{align}
However, equation (5) only applies where both probabilities exist.

Now we can begin to plug in values. Setting $p = 0.6$, $c = b = 5$ and
simplifying a bit, we get the matrix
\[
  P =
\begin{pmatrix}
  \frac{9}{10} & \frac{1}{10} & 0 & 0 & 0 & 0 \\[6pt]
  \frac{11}{15} & \frac{1}{6} & \frac{1}{10} & 0 & 0 & 0 \\[6pt]
  \frac{17}{30} & \frac{1}{6} & \frac{1}{6} & \frac{1}{10} & 0 & 0 \\[6pt]
  \frac{2}{5} & \frac{1}{6} & \frac{1}{6} & \frac{1}{6} & \frac{1}{10} & 0
  \\[6pt]
  \frac{7}{30} & \frac{1}{6} & \frac{1}{6} & \frac{1}{6} & \frac{1}{6} &
  \frac{1}{10} \\[6pt]
  \frac{1}{6} & \frac{1}{6} & \frac{1}{6} & \frac{1}{6} & \frac{1}{6} &
  \frac{1}{6} \\[6pt]
\end{pmatrix}
\]

Solving the equation $(I - P')\pi = 0$ was made easy thanks to the function
\verb findpi1()  from the book. Plugging our matrix into the function gave us
the vector
\[
  \pi =
  \begin{pmatrix}
    0.877 \\
    0.108 \\
    0.0134 \\
    0.00165 \\
    0.000203 \\
    0.0000243 \\
  \end{pmatrix}
\]

To verify the correctness of our answer, we also found that $\sum\limits_i
\pi_i = 1$.

\subsection*{(b)}

Finding the long-run average number of packets which are discarded is trivial,
now that we have our stationary probabilities. We merely need to find the
probability that the buffer is full and one more packet is generated. This is
\begin{align*}
  \pi_bp &= \pi_5\cdot 0.6 \\
  &= 2.43\times 10^{-5}\cdot 0.6 \\
  &= 1.46\times 10^{-5}
\end{align*}

\end{document}
