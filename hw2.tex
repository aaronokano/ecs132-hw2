\documentclass{article}
\usepackage{amsmath, amsthm, amssymb}
\usepackage{listings}
\usepackage{graphicx}
\usepackage{float}
\usepackage{enumerate}
\usepackage{fancyhdr}
\usepackage[labelfont=bf]{caption}
\usepackage[left=0.75in, top=1in, right=0.75in, bottom=1in]{geometry}
\pagestyle{plain}
\begin{document}
\rhead{Aaron Okano, Anatoly Torchinsky, Samuel Huang, Justin Maple \\ 
      ECS 132: Homework 2}
\thispagestyle{fancy}

% Let the homework begin!
\section*{Problem 1} 

\subsection*{(a)}

Since this is not an infinite series, we cannot use identities derived in the
book. Instead, we must do our own. We must solve for $c$ in
$\displaystyle\sum\limits_{k=1}^{10} ck^2 = 1.0$.
\begin{eqnarray*}
  \sum\limits_{k=1}^{10} ck^2 &=& c \cdot \frac{10(10 + 1)(2\cdot10+1)}{6} \\
  &=& c\cdot 385 \\
  &\Rightarrow& c = \frac{1}{385}
\end{eqnarray*}

\subsection*{(b)}

We can find the expected value of X by computing $\sum\limits_{k=1}^{10} k\cdot
p_{X}(k)$. We know $p_{X}(k) = c\cdot k^2$ and $c = \frac{1}{385}$ so,
\begin{eqnarray*}
  \frac{1}{385}\cdot\sum\limits_{k=1}^{10} k\cdot k^2 &=& \frac{1}{385}
  \sum\limits_{k=1}^{10} k^3 \\
  &=& \frac{1}{385} \cdot \left[ \frac{1}{4}\cdot10^4 + \frac{1}{2}\cdot10^3 +
  \frac{1}{4}\cdot10^2\right] \\
  &\approx& 7.86
\end{eqnarray*}
By (3.30), we can compute the variance of X by finding
\begin{eqnarray*}
  \left[\sum\limits_{k=1}^{10} k^2\cdot p_{X}(k) \right] - [EX]^2 &=&
  \frac{1}{385}\cdot \sum\limits_{k=1}^{10} k^3 - 7.86^2 \\
  &=& \frac{1}{385}\left[ \frac{1}{5}\cdot10^5 + \frac{1}{2}\cdot10^4 +
  \frac{1}{3}\cdot10^3 - \frac{1}{30}\cdot10\right] - 7.86^2 \\
  &\approx& 4.02
\end{eqnarray*}

\section*{Problem 2}

First, we take note that each set of coin tosses follows a binomial
distribution. Therefore, we can express $P( X_2 = i ), i = 0,1,2$ as $P( X_2 =
i | C_1 \cup X_2 = i | C_2 ) = P( X_2 = i | C_1 ) + P( X_2 = i | C_2 ) = P(
C_1) P( X_2 = i | C_1 ) + P( C_2 )P( X_2 = i | C_2 )$. This follows
intuitively, since there are two possibilities: one where you pick the
head-weighted coin and one where you pick the tail-weigted coin, after which
the set of tosses is modeled with a binomial distribution. Using this
information, we can construct a pmf for the coin tosses:
\begin{equation*}
  p_{X_2}(k) = 0.5\cdot\binom{2}{k}(0.9)^k(0.1)^{2 - k} +
  0.5\cdot\binom{2}{k}(0.1)^k(0.9)^{2 - k}
\end{equation*}
Plugging this into the handy formula $EX_2 = \sum\limits_{k=0}^{k=2} k\cdot
p_{X_2}(k)$ yields
\begin{eqnarray*}
  EX_2 &=& \sum\limits_{k=0}^{k=2}
  k\cdot\left[0.5\cdot\binom{2}{k}(0.9)^k(0.1)^{2 - k} +
  0.5\cdot\binom{2}{k}(0.1)^k(0.9)^{2 - k}\right] \\
  &=& 0 + \left[(0.5)(2)(0.9)(0.1) + (0.5)(2)(0.1)(0.9)\right] +
  2\cdot\left[(0.5)(0.9)^2 + (0.5)(0.1)^2\right] \\
  &=& 0.18 + 0.82 \\
  &=& 1
\end{eqnarray*}
Finding variance is trivial at this point, since the final term of $EX_2$ can
simply be doubled to give us $E({X_2}^2) = 0.18 + 1.64 = 1.82$. Then $Var(X_2) =
E({X_2}^2) - (EX)^2 = 1.82 - 1^2 = 0.82$.

\end{document}
